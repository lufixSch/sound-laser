\chapter{Concept}\label{sec:concept}

% \begin{enumerate}
%   \item Ultrasonic beamforming
%   \subitem Reason:
%   \subsubitem Smaller speaker
%   \subsubitem Better focus
%   \item Hexagon module
%   \subitem Reason:
%   \subsubitem Extendable/Combinable with multiple modules
%   \subitem Amplification for each module
%   \subitem digital input
%   \item Aiming with physical contraption
%   \subitem no phase shifting necessary
%   \subsubitem to much amplifier circuits
%   \subsubitem to much digital processing -> ?proof?
% \end{enumerate}

The basic idea of this project is to design a small module which can be combined to a larger speaker-array with improved beamforming characteristics. This will give the opportunity to compare different arrangements and numbers of modules. A single module is hexagon-shaped in order to be arranged in a honeycomb like style. Each module is equipped with an onboard DAC and amplification circuit.\p
Because of their size and their more focused dispersion characteristics, ultrasonic transducers are used in this project. How an audible sound can played over an ultrasonic transducer will be explained in section \secref{sec:theory:mod}.\p
The direction of the beam is not controlled by phase shifting because every transducer would need it's own DAC and amplifier. Alternatively the direction is altered by a mechanical contraption.