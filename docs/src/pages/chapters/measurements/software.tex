\section{Software}

\subsection{Measurement Setup}

The validation of the software was done using the \lstcpp{Tabel} class. This class provides multiple vectors wich can be used to save samples or other data. When the program is exited this data is written to stdout using the \textit{CSV} format. This can then be analyzed directly in the terminal or printed to a file and imported into a python notebook.\p
%
Instead of sending the resulting signal over SPI a timestamp is generated and both (timestamp and signal) are added to the table with \lstcpp{addSignal()}. Additionally the duration of one iteration (with sleep) is saved using \lstcpp{addDiff()}. Finally the position in the table is increased to the next row with \lstcpp{nexSignal()}.

\subsection{Discussion}

Figure \dots shows the duration of one iteration over the transmission of a $1s$ signal. It can be observed that the loop does not reach a stable duration. The smaller variations can be explained by the scheduling of linux and should not be a problem. However the graph also shows larger deviations from the intended $10\mu s$ loop duration. \dots.
%
In Figure \dots and \dots the frequency spectrum of a AM and FM modulated sine wave generated by the sound laser software is compared against it's expected frequency spectrum. \dots.
%
\subsection{Audiotest}

In addition to the measurements above the software was tested using the speaker module. Because the original DAC did not work (See section \ref{sec:meas:circuit:dac}) an 8 Bit DAC (\dots) was used. Different sine waves as well as a song where played. In the resulting audio all sounds could be recognized. Unfortunately, the signal was very noisy. As analyzed before in figure \dots the signal also paused repeatedly for a short time.