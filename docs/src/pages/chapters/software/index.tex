\chapter{Software}

The software of the speaker has two main goals: Controlling the movement and playing an audio signal from an external source over SPI. Because both parts of the software don't need to interact with each other it was decided to split them up into two separate programs. The Speaker control is written in python but the Sound playback is written in C++ to increase performance.\p
%
A Raspberry Pi 3 Model B+ was chosen to run the software. With it's Linux operating system it provides an easy way to parallize tasks. The Raspberry Pi also has a SPI interface and enough PWM pins to control the Servo Motors.\cite{van_loo_bcm2836_2014}

\section{Sound playback}

The sound playback program starts by reading samples from a connected audio device. Those samples are then modulated and sent over spi with the right samplingrate. In order to generate a fluent audio playback the tasks where splitted into three threads. The whole sequence of the program is shown in figure \ref{fig:software:sound_sequence}.\p
%
Reading audio samples and modulating them is done in the class \lstcpp{AudioProcessor}. The SPI transmission is handled in the class \lstcpp{Speaker}. Both classes are built with the singleton pattern.
%
\begin{figure}
  \centering
  \includegraphics[width=\textwidth]{src/assets/pictures/software/sequence_diagramm.png}
  \caption{Sound playback sequence diagram}\label{fig:software:sound_sequence}
\end{figure}
\p
The communication between threads is handlend with a queue. The implementation shown in listing \ref{lst:software:queue} makes the \lstcpp{std::queue} threadsafe by using a lock to access the underlying queue and also blocks the thread if no data is available. This is usefull, among other things, because there is no need to send data over SPI when no new sample is generated.\p
%
A hardware abstraction layer is used to separate logic and hardware interfaces.
%
\begin{mdframed}
\begin{lstlisting}[caption=Threadsafe and blocking queue, label=lst:software:queue]
template <class T> class BlockingQueue : public queue<T> {
  public:
  void push(T item) {
    {
      unique_lock<std::mutex> lck(lock);
      queue<T>::push(item);
    }
    not_empty.notify_one();
  }

  T pop() {
    unique_lock<std::mutex> lck(lock);
    not_empty.wait(lck, [this]() { return queue<T>::size() > 0; });

    T value = queue<T>::front();
    queue<T>::pop();
    return value;
  }

  bool notEmpty() { return !queue<T>::empty(); }

  private:
  std::mutex lock;
  condition_variable not_empty;
};
\end{lstlisting}
\end{mdframed}

\subsection{Audio Input}\label{sec:software:rec}
%
There are different methods to connect an external audio device to the Raspberry Pi. The Raspberry provides an AUX port where the microphone line could be used as input. An alternative would be to use an external USB audio card or an audio card HAT\footnote{Hardware attached on top} with a dedicated AUX IN port. The third variant would be to connect the device over bluetooth.\p
%
Because of the current shortage off Raspberry Pi's and Raspberry Pi accessories a HAT is not an option. In order to use the Raspberry as a bluetooth speaker some complicated configuration and programming of the bluetooth interface has to be done, which would take a lot of time. The integrated audio interface of the Raspberry Pi could be used but it is known to have a mediocre sound quality. Therefore the best option would be an external USB audio card. However during research the open source tool BlueZ-Alsa\cite{bokowy_bluez-alsa_2022} was found. This program makes it possible to configure the Raspberry Pi as bluetooth audio speaker with one simple shell-command. The connected device (Audio in/out depending on the configuration) can then be accessed using the Advanced Linux Sound Architecture (ALSA).
%
\subsubsection*{ALSA Interface}

The interface to ALSA is the HAL class \lstcpp{PCM}. On construction it takes a device name, channel count and sampling rate. With those parameters the ALSA PCM device is configured using the \textbf{ALSA} C library (Listing \ref{lst:software:audio_in:pcm_config}).\cite{noauthor_alsa_nodate} The configuration is based on an example of the ALSA Audio API tutorial from Paul Davis.\cite{davis_tutorial_2002}\p
%
\begin{mdframed}
\begin{lstlisting}[caption=ALSA interface configuration, label=lst:software:audio_in:pcm_config]
PCM::PCM(std::string device_name, char channel_cnt, uint32_t fs) {
  int i;
  int err;
  snd_pcm_format_t format = SND_PCM_FORMAT_S16_LE;

  if ((err = snd_pcm_open(&capture_handle, device_name.c_str(), SND_PCM_STREAM_CAPTURE, 0)) < 0) {
    std::cerr << "cannot open audio device (" << snd_strerror(err) << ")\n";
    exit(1);
  }

  if ((err = snd_pcm_hw_params_malloc(&hw_params)) < 0) {
    std::cerr << "cannot allocate hardware parameter structure (" << snd_strerror(err) << ")\n";
    exit(1);
  }

  if ((err = snd_pcm_hw_params_any(capture_handle, hw_params)) < 0) {
    std::cerr << "cannot initialize hardware parameter structure (" << snd_strerror(err) << ")\n";
    exit(1);
  }

  if ((err = snd_pcm_hw_params_set_access(capture_handle, hw_params, SND_PCM_ACCESS_RW_INTERLEAVED)) < 0) {
    std::cerr << "cannot set access type (" << snd_strerror(err) << ")\n";
    exit(1);
  }

  if ((err = snd_pcm_hw_params_set_format(capture_handle, hw_params, format)) < 0) {
    std::cerr << "cannot set sample format (" << snd_strerror(err) << ")\n";
    exit(1);
  }

  if ((err = snd_pcm_hw_params_set_rate(capture_handle, hw_params, (unsigned int)fs, 0)) < 0) {
    std::cerr << "cannot set sample rate (" << snd_strerror(err) << ")\n";
    exit(1);
  }

  if ((err = snd_pcm_hw_params_set_channels(capture_handle, hw_params, channel_cnt)) < 0) {
    std::cerr << "cannot set channel count (" << snd_strerror(err) << ")\n";
    exit(1);
  }

  if ((err = snd_pcm_hw_params(capture_handle, hw_params)) < 0) {
    std::cerr << "cannot set parameters (" << snd_strerror(err) << ")\n";
    exit(1);
  }

  snd_pcm_hw_params_free(hw_params);

  if ((err = snd_pcm_prepare(capture_handle)) < 0) {
    std::cerr << "cannot prepare audio interface for use (" << snd_strerror(err) << ")\n";
    exit(1);
  }

  format_width = snd_pcm_format_width(format) / 8;
  format_zero = 0;
  format_max = pow(2, snd_pcm_format_width(format)) / 2;
}
\end{lstlisting}
\end{mdframed}
%
After the configuration the method \lstcpp{readFrames()} can be used to read a given number of samples (Listing \ref{lst:software:audio_in:pcm_read}). The method will block the code execution until all frames are received or an error occurs.
%
\begin{mdframed}
\begin{lstlisting}[caption=Method for reading frames from an ALSA device, label=lst:software:audio_in:pcm_read]
void PCM::readFrames(sample_t* buffer, size_t frames) {
  int err;

  if ((err = snd_pcm_readi(capture_handle, buffer, frames)) != frames) {
      std::cerr << "read from audio interface failed (" << snd_strerror(err) << ")\n";
      exit(1);
    }
}
  \end{lstlisting}
\end{mdframed}
%
\subsubsection*{Record Audio}
%
The \lstcpp{PCM} class is initialized in \lstcpp{AudioProcessor::configure()}. This instance is then used in the method \lstcpp{record()} shown in listing \ref{lst:software:audio_in:record}. The method creates a loop in which a set of samples is read into a buffer and then pushed into the \lstcpp{AudioProcessor::samples} queue. \lstcpp{record()} can be executed directly or by calling \lstcpp{record_thread()} which creates a new thread for the loop.

\begin{mdframed}
\begin{lstlisting}[caption=Record loop, label=lst:software:audio_in:record]
void AudioProcessor::record() {
  std::cout << "Run Record Loop\n";
  size_t len = frame_size;

  std::vector<sample_t> buffer(len);

  while (true) {
    pcm_dev->readFrames(buffer.data(), frame_size);
    samples.push(buffer);
  }
}
\end{lstlisting}
\end{mdframed}
\section{Modulation}\label{sec:theory:mod}

%\begin{enumerate}
%  \item Demodulation
%  \item Differences
%  \subitem bandwith
%  \subitem noise/disturbance
%  \item Bandwith
%\end{enumerate}
%
Modulation in electronics describes the process of altering parameters of a carrier wave depending on an input signal. This will transfer the frequency of the input signal so it can be transmitted for example via radio wave. The receiver can then demodulate the signal to retrieve the original.

\subsection{Modulation Techniques}

\subsubsection*{AM}
With the amplitude modulation the amplitude of the carrier signal $c(t)$ is changed depending on the modulating signal $s(t)$. The modulation depth $m$ defines to which grade the amplitude is modified. With the parameter $U_0$ the offset of the modulating signal can be changed.\p
The AM is generally defined as follows:
%
\begin{align}
  AM(t) &= U_0 \cdot [1 + m \cdot s(t)] \cdot c(t)
\end{align}
%
Usually the carrier signal is a sine or cosine wave with the carrier frequency $f_c$.
%
\begin{align}
  AM(t) &= U_0 \cdot [1 + m \cdot s(t)] \cdot \cos (2 \pi \cdot f_c \cdot t)\label{eq:theory:mod:am}
\end{align}
%
\subsubsection*{FM}
%
Using frequency modulation the frequency of the carrier signal $c(t, f)$ is changed by the modulating signal $s(t)$. The modulation depth $m$ defines to which grade the frequency is modified. With the parameter $U_0$ the amplitude of the carrier is defined.\p
The FM is generally defined as follows:
%
\begin{align}
  FM(t) &= U_0 \cdot c(t, 2 \pi \cdot f_c \cdot [1 + m \cdot s(t)])\pmath
  FM(t) &= U_0 \cdot \cos (2 \pi \cdot f_c \cdot [1 + m \cdot s(t)] \cdot t) \label{eq:theory:mod:fm}
\end{align}

\subsection{Demodulation}

In order to recreate audible sound from the modulated signal it has to be demodulated. In the case of a parametric array non-linear behaviour of the air leads to the demodulation of AM as well as FM signals.\p
%
In linear acoustics it is presumed, that sound waves don't interact with each other but just superimpose on each. However in reality, when two sound waves collide some local temperaturechanges occur resulting in an inhomogene transmissionmedium. Therefore if two waves spread into the same direction their propagation is affected by each other. The resulting wave is described as an inhomogeneous wave equation by Westervelt (Equation \ref{eq:theory:mod:westervelt}).\cite{westervelt_parametric_1963}\cite{yoneyama_audio_1983}
%
\begin{align}
  \triangledown^2 p_s - \frac{1}{c_0^2} \cdot \frac{\partial^2 p_s}{\partial t^2} & = - p_0 \frac{\partial q}{\partial t}\label{eq:theory:mod:westervelt}\pmath
  q &= \frac{\beta}{\rho_0^2 c_0^4} \cdot \cfrac{\partial }{\partial t} p_1^2 \label{eq:theory:mod:westervelt_q}
\end{align}
%
$p_1$ and $p_s$ are the sound pressure of the the primary and secondary wave, $\beta$ is the nonlinear fluid parameter and $c_0$ is the sound velocity.

\subsubsection*{AM}

Considering equation \ref{eq:theory:mod:am} the sound pressure of an AM signal can be described as shown in equation \ref{eq:theory:mod:am_pressure} and \ref{eq:theory:mod:am_pressure_radius}.
%
\begin{align}
  p_1(t) &= p_0 \cdot (1 + m \cdot s(t)) \cdot \cos (2 \pi \cdot f_c \cdot t)\label{eq:theory:mod:am_pressure}\pmath
  p_1(t, r) &= p_0 e^{- \alpha r} \cdot \left[1 + m \cdot s\left(t - \frac{r}{c_0}\right)\right] \cdot \cos \left(2 \pi \cdot f_c \cdot \left[t - \frac{r}{c_0}\right]\right)\label{eq:theory:mod:am_pressure_radius}
\end{align}
%
$p_0$ is the initial pressure, $r$ is the distance from the speaker and $e^{- \alpha r}$ describes the damping of the wave over this distance.\p
$p_1$ is now inserted into equation \ref{eq:theory:mod:westervelt_q}.
%
\begin{align}
  q &= \frac{\beta p_0^2}{\rho_0^2 c_0^4} e^{ - 2 \alpha r} \cdot \cfrac{\partial }{\partial t} \left[1 + m \cdot s\left(t - \frac{r}{c_0}\right)\right]^2 \cdot \cos^2 \left(2 \pi \cdot f_c \cdot \left[t - \frac{r}{c_0}\right]\right)\pmath
  &\mathrm{with~} \cos^2(\varphi) = \frac{1}{2} [1 + \cos (2 \varphi)]\pmath
  q &= \frac{\beta p_0^2}{\rho_0^2 c_0^4} e^{ - 2 \alpha r} \cdot \cfrac{\partial }{\partial t} \left[1 + m \cdot s\left(t - \frac{r}{c_0}\right)\right]^2 \cdot \left[1 + \cos \left(4 \pi \cdot f_c \cdot \left[t - \frac{r}{c_0}\right]\right)\right]
\end{align}
%
As the focus of the demodulation is audible sound all signals in the ultrasonic range are negligible.
%
\begin{align}
  q &= \frac{\beta p_0^2}{\rho_0^2 c_0^4} e^{ - 2 \alpha r} \cdot \cfrac{\partial }{\partial t} \left[1 + m \cdot s\left(t - \frac{r}{c_0}\right)\right]^2
\end{align}
%
If this solution is now inserted into equation \ref{eq:theory:mod:westervelt} it can be solved for $p_s$. The result is known under the \cite[Berktay far-field solution]{bai_analysis_2012}
%
\begin{align}
  p_s(t) = \frac{\beta p_0^2 S}{16 \pi \rho_0^2 c_0^4 \alpha_0 r} \frac{\partial^2}{\partial t^2} E^2(t - \frac{r}{c_0})\label{eq:theory:mod:berktay}
\end{align}
%
Where $S$ is the area of the parametric array and $E(t - \cfrac{r}{c_0})$ is the function envelope of $p_1(t,r)$.\p
Equation \ref{eq:theory:mod:am_result} shows the demodulated AM signal using the Berktay far-field solution. The demodulation results not only in the original signal but distortion from the interaction between lower and upper sideband. However with a modulation depth $m < 1$ the distortion will always be smaller than the signal.
%
\begin{align}
  p_s(t) = \frac{\beta p_0^2 S}{16 \pi \rho_0^2 c_0^4 \alpha_0 r} \frac{\partial^2}{\partial t^2} \left[ 2 m \cdot s\left(t - \frac{r}{c_0}\right) + m^2 \cdot s^2\left(t - \frac{r}{c_0}\right)\right]\label{eq:theory:mod:am_result}
\end{align}
%

%This equation can now be inserted into equation \ref{eq:theory:mod:westervelt_q}.
%
%\begin{align}
%  q &= \frac{\beta \cdot p_0^2}{\rho_0^2 c_0^4 } e^{- 2 \alpha r} \cdot \cfrac{\partial }{\partial t} \left(\left[1 + m \cdot s\left(t - \frac{r}{c_0}\right)\right]^2 \cdot \cos^2 \left(2 \pi \cdot f_c \cdot \left[t - \frac{r}{c_0}\right]\right)\right)\pmath
%  \mathrm{with\space}\cos^2(x) &= \frac{1}{2} \cos (2x) + 1\pmath
%  q &= \frac{\beta \cdot p_0^2}{\rho_0^2 c_0^4 } e^{- 2 \alpha r} \cdot \cfrac{\partial }{\partial t} \left(\left[1 + m \cdot s\left(t - \frac{r}{c_0}\right)\right]^2 \cdot \frac{1}{2} \left[ 1 + \cos \left(4 \pi \cdot f_c \cdot \left[t - \frac{r}{c_0}\right]\right)\right]\right)
%\end{align}
%
%Because of a damping effect of $12dB/oct$ the cosine is negligible leaving the original signal and some distortion. But with a modulation depth $m < 1$ the distortion will always be smaller than the signal.
%
%\begin{align}
%  q &= \frac{\beta \cdot p_0^2}{\rho_0^2 c_0^4 } e^{- 2 \alpha r} \cdot \cfrac{\partial }{\partial t} \left[ m \cdot s\left(t - \frac{r}{c_0}\right) + \frac{1}{2} m^2 \cdot s^2\left(t - \frac{r}{c_0}\right)\right]
%\end{align}
%
%Using equation \ref{eq:theory:mod:weservelt} the sound pressure of the secondary wave can be described as following:
%
%\begin{align}

%\end{align}
%

\subsubsection*{FM}

Using the same approach as with the AM signal, the demodulation of the FM signal can be calculated as well. Having said this the process is mostly ignored when talking about parametric speaker arrays and therefore information about it is quite rare. Usually the effect is only verified empirically. The paper of Masato Nakayama and Takanobu Nishiura gives an idea on how the mathematical solution might look.\cite{nakayama_synchronized_2017}

%As with the AM the sound pressure of the FM signal is described using equation \ref{eq:theory:mod:fm}.
%%
%\begin{align}
%  p_1(t) &= p_0 \cdot \cos (2 \pi \cdot f_c \cdot [1 + m \cdot s(t)] \cdot t)\pmath
%  p_1(t, r) &= p_0 e^{- \alpha r} \cdot \cos (2 \pi \cdot f_c \cdot [1 + m \cdot s(t)] \cdot t)
%\end{align}
%%
%Using the relation between phase and frequency the FM signal is modified.
%%
%\begin{align}
% \int \omega(t) &= \varphi(t)\pmath
% 2 \pi \cdot f_c \cdot [1 + m \cdot s(t)] &= 2 \pi \cdot f_c t + \int m \cdot s(t)\pmath
% \implies p_1(t, r) &= p_0 e^{- \alpha r} \cdot \cos (2 \pi \cdot f_c t + \int m \cdot s(t))
%\end{align}
%%
%%This equation is than inserted into equation \ref{eq:theory:mod:westervelt_q}
%%
%%\begin{align}
%%  q &= \frac{\beta \cdot p_0^2}{\rho_0^2 c_0^4 } e^{- 2 \alpha r} \cdot \cfrac{\partial }{\partial t} \left( \cos^2 (2 \pi \cdot f_c \cdot [1 + m \cdot s(t)] \cdot t) \right)\pmath
%%  \mathrm{with\space}\cos^2(x) &= \frac{1}{2} \cos (2x) + 1\pmath
%%  q &= \frac{\beta \cdot p_0^2}{\rho_0^2 c_0^4 } e^{- 2 \alpha r} \cdot \cfrac{\partial }{\partial t} \cdot \frac{1}{2} \left(1 + \cos (4 \pi \cdot f_c \cdot [1 + m \cdot s(t)] \cdot t) \right)\pmath
%%\end{align}
%%
%Assume:
%%
%\begin{align}
%  g(t) =& \int m \cdot s(t)\pmath
%  E(t) =& \cos^2 (2 \pi \cdot f_c t + g(t))
%\end{align}
%%
%Now the signal $E(t)$ can be inserted into the Berktay far-field solution (Equation \ref{eq:theory:mod:berktay}).
%%
%\begin{align}
%  p_s(t) =& \frac{\beta p_0^2 S}{16 \pi \rho_0^2 c_0^4 \alpha_0 r} \frac{\partial^2}{\partial t^2} \cos^2 (2 \pi \cdot f_c t + g(t))
%  %p_s(t) =& \frac{\beta p_0^2 S}{16 \pi \rho_0^2 c_0^4 \alpha_0 r} \left[\sin^2 (2 \pi \cdot f_c t + G(t)) (2 \pi \cdot f_c + g'(t))^2\right. \pmath
%  %&- \cos^2 (2 \pi \cdot f_c t + g(t)) (2 \pi \cdot f_c + g'(t))^2 \pmath
%  %&\left.- \cos (2 \pi \cdot f_c t + g(t)) \sin (2 \pi \cdot f_c t + g(t)) g''(t)\right] \pmath
%\end{align}
%%
%Next the signal is differentiated.
%%
%\begin{align}
%  \frac{\partial }{\partial t} E(t) =& - 2\cos (2 \pi \cdot f_c t + g(t)) \sin (2 \pi \cdot f_c t + g(t)) (2 \pi \cdot f_c + g'(t)) \pmath
%  \frac{\partial^2 }{\partial t^2} E(t) =& 2\sin^2 (2 \pi \cdot f_c t + G(t)) (2 \pi \cdot f_c + g'(t))^2 \pmath
%  &- 2\cos^2 (2 \pi \cdot f_c t + g(t)) (2 \pi \cdot f_c + g'(t))^2 \pmath
%  &- 2\cos (2 \pi \cdot f_c t + g(t)) \sin (2 \pi \cdot f_c t + g(t)) g''(t) \pmath
%  %
%  \mathrm{with\space} \cos^2(\varphi) &= \frac{1}{2} [1 + \cos (2\varphi)] \mathrm{\space and \space} \sin^2(\varphi) = \frac{1}{2} [1 - \cos (2\varphi)]\pmath
%  %
%  \frac{\partial^2 }{\partial t^2} E(t) =& [1 - \cos(4 \pi \cdot f_c t + G(t)] [2 \pi \cdot f_c + g'(t)]^2 \pmath
%  &- [1 + \cos(4 \pi \cdot f_c t + g(t)][2 \pi \cdot f_c + g'(t)]^2 \pmath
%  &- \cos(2 \pi \cdot f_c t + g(t)) \sin (2 \pi \cdot f_c t + g(t)) g''(t) \pmath
%\end{align}
%%
%As the focus of the demodulation is audible sound, all ultrasonic frequencies are negligible, leaving only the first derivative $g'(t)$.
%%
%\begin{align}
%  \frac{\partial^2 }{\partial t^2} E(t) =& [1 - \cos(4 \pi \cdot f_c t + G(t)] [2 \pi \cdot f_c + g'(t)]^2 \pmath
%  &- [1 + \cos(4 \pi \cdot f_c t + g(t)][2 \pi \cdot f_c + g'(t)]^2 \pmath
%  &- \cos(2 \pi \cdot f_c t + g(t)) \sin (2 \pi \cdot f_c t + g(t)) g''(t) \pmath
%\end{align}
\subsection{Bandwith}

\subsubsection*{AM}

The bandwith calculation of an amplitude modulated signal is straightforward as the modulation just creates an upper and lower sideband at $f_c \pm f_s$.\cite{netzberger_kommunikationstechnologie_2021-1} Where $f_c$ is the carrier frequency and $f_s$ is the signal frequency.\p
As the highest tone a human can hear lies at about $20kHz$, this would result in a maximum bandwith of $f_c \pm 20kHz$. With a carrier frequency of $40kHz$ the highest frequency of the modulated signal would lie at $60kHz$. Reducing the highest tone to $8kHz$, which is enough for most audio would reduce this value to $48kHz$.

\subsubsection*{FM}

In theory the frequency spectrum of a frequency modulated signal is infinite with decreasing pulses at $f_c \pm n \cdot f_s$. However the bandwith can be limited without causing significant damage to the original signal.\cite{netzberger_kommunikationstechnologie_2021-1}
%
\begin{align}
  B =& 2 (\triangle f + f_s) &\textrm{with~} \triangle f = m \cdot f_c
\end{align}
%
With a modulation depth of $0.05$, a carrier frequency of $40kHz$ and a maximum signal frequency of $20kHz$ this would result in a bandwith of $44kHz$. With a maximum signal frequency of $8kHz$ this would be reduced to $20kHz$. The highest frequency of the modulated signal would therefore be $50kHz$.
\subsection{SPI (Audio output)}

\begin{enumerate}
  \item SPI samplingrate issue
  \subitem solution
  \item Send SPI with samplingrate
\end{enumerate}

\section{Speaker movement}
\subsection{Servo Motors}

\subsection{Control interface}