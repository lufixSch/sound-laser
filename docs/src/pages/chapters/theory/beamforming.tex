\section{Beamforming}

\begin{enumerate}
%  \item Explanation
%  \subitem straight line
%  \subitem phase shifting
%  \item Ideal calculation
%  \subitem Number of speakers
%  \subitem Distance of speakers
  \item Real-World behavior/limitations
  \subitem Real speaker distance
  \subitem Real speaker characteristics
\end{enumerate}

Beamforming definition \dots

In order to produce the beamforming effect you need multiple emitters arranged in one line.
If the receiver is far enough away from the emitters the ? directional vector ? of the waves can be viewed as parallel (See figure \dots). When the receiver is positioned directly in front of the emitters (\(0^\circ\)) all waves reach it at the same time (See figure \dots).
When the position of the receiver is rotated (\(\pm x^\circ\)) the waves have to travel different distances. Therefore they reach the receiver with a phase shift (See figure \dots). Depending of the position of the receiver and distance of the emitters the waves will cancel each other out.

The difference in travel distance of two waves depending on the position \(\varphi\) can be described as follows. \(d\) describes the spacing between the emitters.

\begin{align}
  l(\varphi) &= \sin \varphi \cdot d
\end{align}

Given the connection between distance, time and velocity. This equation can be used to calculate the phaseshift between those two waves.

\begin{align}
  t(\varphi)     &= \frac{l}{v} = \frac{d}{v} \sin \varphi \\[1em]
  \Phi(\varphi)  &= \omega \cdot t = \frac{\omega}{v} d \cdot \sin \varphi \\[1em]
  \mathrm{with~} \lambda &= \frac{v}{f} \implies \frac{\omega}{v} = \frac{2\pi \cdot f}{v} = \frac{2\pi}{\lambda} \\[1em]
  \Phi(\varphi)  &= \frac{2\pi}{\lambda} d \cdot \sin \varphi
\end{align}

This can be extended for a whole array of emitters just by summing up the phaseshift.

\begin{align}
  \Phi(\varphi, n) &= n \cdot \frac{2\pi}{\lambda} d \cdot \sin \varphi
\end{align}

Now the phaseshift of each emitter can be calculated relative to the first on (\(n = 0\)).

Given a set of two spherical emitter generating a cosine wave (\(s_e(t)\)) the received signal (\(s(t)\)) would look as shown in figure \dots.

\begin{align}
  s_e(t, \varphi, n) = \cos (\omega t + n \cdot \frac{2\pi}{\lambda} d \cdot \sin \varphi )\\[1em]
  s(t, \varphi) = \sum_{n = 0}^{N} s_e(t, \varphi, n)
\end{align}

It can be observed, that the amplitude of the resulting signal is reduced the larger \(\varphi\) gets. Until both waves cancel each other out at \(90^\circ\).

For a better analysis the resulting signal \(s(t, \varphi)\) is fourier transformed. Now the amplitude of the signal can be displayed depending on the position \(\varphi\) in a polar graph. The Amplitude is displayed in dB relative to the highest amplitude (See figure \dots).