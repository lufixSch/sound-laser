\section{Tilting mechanism}

\begin{enumerate}
  \item Requirements
  \subitem Tilting in two dimensions
  \subitem Remove Stress from Motors
  \subitem Angular range
  \item Design -> Reference to Design
  \item Servo Position to tilting angle -> Calculation
  \subitem approximation
\end{enumerate}

\subsubsection*{Concept}

The tilting mechanism should allow the PCBs to tilt around two axis. For the concept the PCBs are displayed as a plane. Figure \dots shows the front and top view of the tilting mechanism for one axis.\p
The PCBs are hold in place by a joint in the center of the plane (\textbf{A}). Now if the top of the plane is moved forwards or backwards (\textbf{B}) the plane rotates. The forward and backward movement can be created by connecting to shafts (\textbf{D} and \textbf{E}) with a joint. The base of \textbf{D} is fixed (\textbf{F}) while the end of \textbf{E} is connected to the top of the plane. This creates the triangle \textbf{D} - \textbf{E} - \textbf{B}. When the base of \textbf{D} is now rotated the shape of the triangle is changed, increasing or decreasing the length of \textbf{B}.
The rotation around two axis can be achieved by adding a second \textbf{D} - \textbf{E} - \textbf{B} triangle to the left or right side of the plane.\p
%


\subsubsection*{Integration}