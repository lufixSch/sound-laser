\section{Powersupply}

\begin{enumerate}
  \item Circuit design
  \subitem In -> Out Voltage
  \subitem Capacitors
  \subitem Inductor
  \item PCB layout
  \subitem Max current strength
  \subsubitem > 2A -> Changed later
  \subitem Parasitic effects/interferences
\end{enumerate}

Because the speaker will be used with a Raspberry Pi or microcontroller, it should work with a input voltage of $5V$. In order to provide the $24V$ needed by the amplifier circuit a boost converter is used.\p
The Class B Amplifier is designed to drive up to $100mA$. Aditionally the OpAmp itself and the circuit around it will need some current. Therefore the powersupply should provide at least $200mA$. Apart from this a higher switching frequency would help to reduce interferences at the speaker output.\p
%
Considering those requirements the \dots boost converter was chosen. Because the power usage of the ultrasonic transducers wasn't clear in the beginning of this project the powersupply provides up to $3A$ output voltage. The output voltage can be adjusted between \dots and \dots.

\subsection{Circuit}


\subsection{PCB Layout}