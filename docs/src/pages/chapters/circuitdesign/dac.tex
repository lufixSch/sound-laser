\section{DAC}
%
%\begin{enumerate}
%  \item Requirements
%  \subitem >= 120ks/s
%  \subitem Fast enough digital interface
%  \subsubitem package size * sampling rate
%  \subitem Handsolder SMD package
%  \subitem 5V VCC
%  \item component/model decision
%  \item I2C vs. SPI
%  \item pcb layout
%  \item Problems with the DAC
%\end{enumerate}
%
The DAC in the Circuit is used to generate an analog voltaged signal from the signal recorded and modulated by a Raspberry Pi or microcontroller.\p
%
As described in section \secref{sec:theory:mod} the frequency of the modulated signal goes up to $50\,kHz$.
In order to comply with the sampling theorem, the sampling frequency of the DAC should be at least $100\,kHz$. Therefore the digital interface has to bee fast enough, even if up to 7 modules are connected at the same time. Additionally, the DAC should work with $5\,V$ supply voltage and an internal reference voltage.\p
%
I2C only allows a maximum clockspeed of $5\,MHz$ (Ultra Fast-mode).\cite{nxp_i2c-bus_2021} The length of a $10\,bit$ or $12\,bit$ DAC sample is usually about $32\,bit$ ($8\,bit$ Address + RW, $8\,bit$ config, $10 - 12\,bit$ Data, $4 - 6\,bit$ Don't care). As shown in equation \ref{eq:pcb:dac_fs} this only allows a sampling rate up to $156.25\,kHz$ even if only one module is connected. Because every module needs to be addressed individually, the sampling rate would drop to $22.321\,kHz$ when using all seven modules. For this reason SPI was selected as serial interface. The clock frequency of the SPI module is only limited by the system clock and the receiver is selected by a separate chip select pin which reduces the packet size of one sample to $24\,bit$. Because every module gets the same signal one CS pin for all DACs can be used.\p
%
%First the 12bit DAC \dots was selected. It uses a I2C interface which allows to connect the Raspberry Pi and the Speaker with only two wires independent of the number of modules. Unfortunately the DAC only supports a clock frequency up to $400kHz$ and each sample has a packetsize of $32bit$. As shown in equation \ref{eq:pcb:fs} this only allows a sampling rate up to $12500Hz$ even if only one module is connected.
%
\begin{align}
  f_{s,1} &= \frac{5MHz}{32bit} = \frac{5 \frac{Mbit}{s}}{32bit} = 156.25kHz \label{eq:pcb:dac_fs}\\[1em]
  f_{s, 7} &= \frac{fs_1}{7} = \frac{156.25kHz}{7} = 22.321kHz
\end{align}
%
The DAC, selected for this project, is the LTC2640 from Linear Technology. It comes with an SPI interface and a sampling rate up to $125\,kHz$. Depending on the model it supports a resolution of $8$, $10$ or $12\,bit$. The DAC has an internal reference voltage of $2.5\,V$ or $4.096\,V$. $12\,bit$ resolution with $4.096\,V$ reference voltage would be the ideal configuration of this DAC. Because of component shortage only the DAC with $2.5\,V$ reference voltage was available. This model produces a smaller output voltage which can be compensated by increasing the gain of the amplifier circuit described in section \secref{sec:pcb:amp}.\cite{linear_technology_ltc2640_2017}
%
\begin{figure}
  \centering
  \includegraphics[height=\mediumheight]{src/assets/pictures/circuit/dac_circuit.png}
  \caption{DAC Circuit design}\label{fig:pcb:dac_circuit}
\end{figure}