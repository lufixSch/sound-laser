\section{Amplifier}\label{sec:pcb:amp}

%\begin{enumerate}
  %\item Requirements
  %\subitem VCC >= 24V -> Why
  %\item pcb layout
%\end{enumerate}

Ultrasonic transducer have to be driven with a much higher voltage and current than a DAC can provide. Therefore an amplifier circuit was designed which takes the DAC voltage as input and generates a suitable output for the array of ultrasonic transducers.

\subsection{Circuit}

The amplifier (Figure \ref{fig:pcb:amp_circuit}) consists of two sections. First a non-inverting amplifier circuit to amplify the input voltage. Next a Class B power amplifier is used so the circuit is able to drive the high capacitive load of the transducers. The whole circuit is designed for single supply because the power supply only provides a positive voltage.

\begin{figure}
  \centering
  \includegraphics[height=\largeheight]{src/assets/pictures/circuit/amp_circuit.png}
  \caption{Amplifier circuit design}\label{fig:pcb:amp_circuit}
\end{figure}

\subsubsection*{Voltage input}
%
\comp{C8} is used to filter the DC part of the input signal. \comp{R8} and \comp{R9} then generate a new operating point at $12V$. For AC signals a DC power supply can be seen as short circuit. This will result in \comp{R8} and \comp{R9} being parallel (Figure \ref{fig:pcb:amp_hp}). Combined with \comp{C8} they build a simple high-pass which can be calculated for a specific cut-off frequency.
%
\begin{align}
  R_8 || R_9 &= \frac{1}{2} R_{8/9} \pmath
  \cfrac{U_a}{U_{in}} &= \cfrac{sC_8\cdot \frac{1}{2}R_{R8/9}}{1 + sC_8\cdot \frac{1}{2}R_{8/9}} \pmath
  f_g &= \cfrac{1}{2 \pi C_8 \cdot \frac{1}{2}R_{R8/9}} \pmath
  C_8 &= \cfrac{1}{2 \pi f_g \cdot \frac{1}{2}R_{R8/9}}
\end{align}
%
As cut-off frequency $4kHz$ was chosen and for \comp{R8} and \comp{R9} a value of $100k\Omega$ was selected. This results in a capacity of around $1.5nF$ for \comp{C8}.
%
\begin{figure}
  \centering
  \includegraphics[height=\smallheight]{src/assets/pictures/circuit/amp_hp_circuit.png}
  \caption{Input hightpass circuit}\label{fig:pcb:amp_hp}
\end{figure}
%
\subsubsection*{Feedback}
%
Because of the new operating point of the input only AC signals should be amplified by the OpAmp. This is achieved with the capacity \comp{C9} in the feedback. It blocks signals below the cut-off frequency, creating a voltage follower. For signals above the cut-off frequency the capacity \comp{C9} acts as a short circuit. The transfer function of the feedback can be used to calculate \comp{C9}. A capacitance of $10nF$ was selected. This results in a cut-off frequency of around $3.2kHz$.
%
\begin{align}
  A &= \frac{1 + s C_9 (R_3 + R_4)}{1 + s C_9 R_4}\pmath
  f_g &= \frac{1}{2 \pi C_9 R_4}
\end{align}
%
The amplification can be calculated like a non-inverting amplifier. In order to take advantage of the $24V$ voltage range with the $4V$ DAC output (At this point it was not clear, that this version of the DAC would not be available) the amplification should be $4$. With $R_3 = 15k\Omega$ this results in \comp{R4} of around $5.1k\Omega$.
%
\begin{align}
  V = 1 + \frac{R_3}{R_4}
\end{align}
%
As described above a power amplifier is needed to drive the high current of the ultrasonic transducers. A class B amplifier was chosen because of its simple but powerefficient design.\cite{okorn_halbleiterschaltung_2020} Usually this design causes crossover distortion but this is solved by putting the feedback behind the power amplifier. Note, this circuit design is only possible if the OpAmp is powerfull enough to drive the base current of the transistors. A better solution for higher currents would be to use the more complex class AB power amplifier.
\subsubsection*{Voltage output}
At the output of the circuit a   filter is used to filter the $12V$ DC voltage caused by the voltage divider at the input.
%
\begin{align}
  A &= \cfrac{sC_{10} \cdot Z_{10}}{1 + sC_{10} \cdot Z_{10}} \pmath
  f_g &= \frac{1}{2 \pi C_{10} \cdot Z_{10}}
  Z_{10} = R_{10} || Z_M
\end{align}
%
Because the behaviour of this high-pass depends on the $ESR$ and capacity of all transducers, the size of \comp{C10} was determined empirically. The best results where achieved with $10uF$.\p

\subsection{Stability}

As mentioned before the feedback of the amplification circuit includes a capacitance. To improve stability OpAmps are usually phase compensated to represent a first order low-pass. Introducing a capacitance into the feedback could disturbed the Stability of the circuit.\cite{okorn_analoge_2021}\p
A general amplification circuit can be represented as shown in figure \ref{fig:pcb:amp:stab:block}. The output signal can therefore be represented as shown in equation \ref{eq:pcb:amp:stab:ua}.
%
\begin{align}
  U_a &= \underbar{A}_d (U_e - \underbar{k}U_a)\pmath
  U_a &= U_e \cfrac{\underbar{A}_d}{1 + \underbar{k}\underbar{A}_d}\label{eq:pcb:amp:stab:ua}
\end{align}
%
\begin{figure}
  \centering
  \includegraphics[width=0.8\textwidth]{src/assets/pictures/circuit/amp_block_circuit.png}
  \caption{Block circuit of an amplification circuit\cite{okorn_analoge_2021}}\label{fig:pcb:amp:stab:block}
\end{figure}
%
This results in the circuit being unstable if $kA_d = -1$.
%
\begin{align}
  |\underbar{k}\underbar{A}_d| &= 1\pmath
  \arg(\underbar{k}\underbar{A}_d) &= - 180^\circ
\end{align}
%
In order to verify the stability, $\underbar{A}_d$ and $\underbar{k}$ for the amplifier circuit where calculated.
%
\begin{align}
  \underbar{A}_d &= \frac{A_0}{1 + \frac{s}{w_g}}\pmath
  w_g &= 2\pi \cdot \frac{GBW}{A_0}
\end{align}
%
The open loop gain $A_0$ as well as the $GBW$ were taken from the data sheet.\cite{texas_instruments_tl08xx_2021}
%
\begin{align}
  \underbar{k} &= \frac{1 + s C_9 R_4}{1 + s C_9 (R_4 + R_3)}
\end{align}
%
Figure \ref{fig:pcb:amp:stab:loop_gain} shows the bode plot of $\underbar{k}\underbar{A}_d$. It can be observed that the amplifier still has a phase reserve of $55^\circ$, which is enough to keep it stable.
%
\begin{figure}
  \centering
  \includegraphics[height=\largeheight]{src/assets/pictures/circuit/amp_stability.pdf}
  \caption{Loop gain}\label{fig:pcb:amp:stab:loop_gain}
\end{figure}
%