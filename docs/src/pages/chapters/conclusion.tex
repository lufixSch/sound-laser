\chapter{Conclusion}

%\begin{enumerate}
%  \item Achievements
%  \item Problems
%  \subitem Hardware
%  \subsubitem DAC
%  \subsubitem Bandwidth
%  \subsubitem Serial interface
%  \subitem Software
%  \subsubitem Modulation
%  \subsubitem Timpe vector will cut the carrier signal
%  \subsubitem Realtime
%  \item Follow up
%  \subitem Larger speaker
%  \subitem Hardware modulation
%  \subitem Communication with uC
%  \subitem Integration with AI Motion Lab
%  \subitem Extended Speaker Control interface
%\end{enumerate}

\subsubsection*{Summary}

In this project it was achieved to created a parametric speaker array which emits audible sound with clear beamforming behaviour. A circuit to operate this speaker was designed and a software was developed which receives audio over bluetooth modulates the signal and transmits it to the speaker. Additionally, a construction to tilt the speaker was built.
%
\subsubsection*{Follow Up}
%
Even though the circuit did not work in the end, this can be easily fixed by replacing the DAC with an other model. The performance problems of the software can be fixed as well by porting the code onto a realtime capable device. A compromise would be to record and modulate the signal on the Raspberry Pi and transmit it to a microcontroller in a large buffer. Due to that, the microcontroller transmits the signal with the given sampling rate.\p
%
Following this project, an in-depth analysis of different modulation techniques, for example pulse modulation or single sideband AM, could be conducted. Furthermore, it would be interesting to examine the behaviour of multiple speaker modules as this would increase the beamforming characteristics.
%
\subsubsection*{Ressources}
%
The source code of the software as well as 3D Modells for the tilting mechanism and a KiCAD project for the PCB design can be accessed on Github: \href{https://github.com/lufixSch/sound-laser}{https://github.com/lufixSch/sound-laser}.